\TitlePage
  \vspace*{55mm}
  \TEXT{\textbf{Palavras-Chave}}
       {\textbf{Visão por Computador, Processamento Natural de Texto, ImageClef, Lifelogging, Recuperação de Momentos}}
  \TEXT{\textbf{Resumo}}
       {Na sociedade moderna, quase qualquer pessoa consegue capturar momentos e registar eventos graças à facilidade de acesso a smartphones. Isso coloca a questão: Se registamos tanto da nossa vida, como podemos facilmente recuperar momentos específicos? A resposta a essa questão abriria a porta para um grande salto na qualidade da vida humana. As possibilidades são infinitas, desde problemas triviais como encontrar a foto de um bolo de aniversário até ser capaz de analisar o progresso de doenças mentais em pacientes ou mesmo rastrear pessoas com doenças infecciosas.}
     \TEXT{}
          {Com tantos dados a serem criados todos os dias, a resposta a esta pergunta tornasse mais complexa. Não existe uma abordagem linear para resolver o problema da localização de momentos num grande conjunto de imagens e pesquisas sobre este problema começaram há apenas poucos anos atrás. O ImageClef é uma competição onde algumas destas pesquisas participam e mostram o seu melhor desempenho na tarefa de recuperação de momentos a cada ano.}
     \TEXT{}
     {Este problema complexo, juntado ao interesse em participar na competição do ImageClef, apresentam-se como um bom desafio para o desenvolvimento desta dissertação.}
     \TEXT{}
     {A solução proposta consiste num sistema capaz de recuperar automaticamente imagens de determinados momentos descritos num formato de texto, usando métodos estado da arte de processamento de imagem e texto de última geração.} 
\EndTitlePage
\titlepage\ \endtitlepage % empty page

\TitlePage
  \vspace*{55mm}
  \TEXT{\textbf{Keywords}}
       {\textbf{Computer Vision, Natural Language Processing, ImageCleff, Lifelogging, Moment Retrieval}}
  \TEXT{\textbf{Abstract}}
       {In our modern society almost anyone is able to capture moments and record events thanks to the ease accessibility to smartphones. This brings the question, if we record so much of our life how can we easily retrieve specific moments? The answer to this question would open the door for a big leap in human life quality. The possibilities are endless, from trivial problems like finding a photo of a birthday cake to being capable of analyzing the progress of mental diseases in patients or even tracking people with infectious diseases. }
  \TEXT{}
       {With so much data being created everyday, answering the question becomes complex. There is no stream lined approach to solve the problem of moment localization in a big dataset of images and researches into this problem have only started a few years ago. ImageClef is one competition where some of these researches participate and show their best performance every year in the task of moment retrieval.}


     \TEXT{}
     {This complex problem, along with the interest in participating in the ImageClef challenge posed a good challenge for the development of this dissertation.}

     \TEXT{}
     {The proposed solution consists in a system capable of automatically image retrieval of specified moments described in a text format using state-of-the-art image and text processing methods.}
\EndTitlePage
\titlepage\ \endtitlepage % empty page
