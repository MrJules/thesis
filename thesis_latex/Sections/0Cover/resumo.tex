\TitlePage
  \vspace*{55mm}
  \TEXT{\textbf{Palavras-Chave}}
       {\textbf{Visão por Computador, Processamento Natural de Texto, ImageClef, Lifelogging, Recuperação de Momentos}}

          
     \TEXT{\textbf{Resumo}}
      {Na sociedade moderna, praticamente qualquer pessoa consegue capturar momentos e registar eventos devido à facilidade de acesso a \textit{smartphones}. Isso leva à questão, se registamos tanto da nossa vida, como podemos facilmente recuperar momentos específicos? A resposta a esta questão abriria a porta para um grande salto na qualidade da vida humana. As possibilidades são infinitas, desde problemas triviais como encontrar a foto de um bolo de aniversário até ser capaz de analisar o progresso de doenças mentais em pacientes ou mesmo rastrear pessoas com doenças infecciosas.}

     \TEXT{}
     {Com tantos dados a serem criados todos os dias, a resposta a esta pergunta torna-se mais complexa. Não existe uma abordagem linear para resolver o problema da localização de momentos num grande conjunto de imagens e investigações sobre este problema começaram há apenas poucos anos. O ImageClef é uma competição onde investigadores participam e tentam alcançar novos e melhores resultados na tarefa de recuperação de momentos a cada ano.}

     \TEXT{}
     {Este problema complexo, em conjunto com o interesse em participar na tarefa LMRT do ImageClef, apresentam-se como um bom desafio para o desenvolvimento desta dissertação.}

     \TEXT{}
     {A solução proposta consiste num sistema capaz de recuperar automaticamente imagens de momentos descritos em formato de texto, sem qualquer tipo de interação de um utilizador, utilizando apenas métodos estado da arte de processamento de imagem e texto.} 

     \TEXT{}
     {O sistema de recuperação desenvolvido alcança este objetivo através da extração e categorização de informação relevante do texto enquanto calcula um valor de similaridade com outros rótulos extraídos da fase de processamento de imagem. Dessa forma, o sistema consegue dizer se as imagens estão relacionadas ao momento especificado no texto e, portanto, é capaz de recuperar as imagens de acordo.
     }

     \TEXT{}
     { Na subtarefa ImageCLEF LMRT 2020, o sistema de recuperação automática proposto alcançou uma pontuação de 0,03 na metodologia de avaliação F1-measure @ 10. Mesmo que estas pontuações não sejam competitivas quando comparadas às pontuações de outros sistemas de outras equipas, o sistema construído apresenta-se como uma boa base para trabalhos futuros.}


\EndTitlePage
\titlepage\ \endtitlepage % empty page

\TitlePage
  \vspace*{55mm}
  \TEXT{\textbf{Keywords}}
       {\textbf{Computer Vision, Natural Language Processing, ImageCleff, Lifelogging, Moment Retrieval}}

     \TEXT{\textbf{Abstract}}
     {In our modern society almost anyone is able to capture moments and record events due to the ease accessibility to smartphones. This leads to the question, if we record so much of our life how can we easily retrieve specific moments? The answer to this question would open the door for a big leap in human life quality. The possibilities are endless, from trivial problems like finding a photo of a birthday cake to being capable of analyzing the progress of mental illnesses in patients or even tracking people with infectious diseases.}

     \TEXT{}
     {With so much data being created everyday, answering the question becomes complex. There is no stream lined approach to solve the problem of moment localization in a large dataset of images of and investigation into this problem have only started a few years ago. ImageClef is one competition where researchers participate and try to achieve new and better results in the task of moment retrieval.}

     \TEXT{}
     {This complex problem, along with the interest in participating in the ImageClef Life Moment Retrieval Task posed a good challenge for the development of this dissertation.}

     \TEXT{}
     {The proposed solution consists in developing a system capable of retriving images automatically  according to specified moments described in a corpous of text without any sort of user interaction using only state-of-the-art image and text processing methods.}

     \TEXT{}
     {The developed retrieval system achieves this objective by extracting and categorizing relevant information from text while being able to compute a similarity score with the extracted labels from the image processing stage. In this way, the system is capable of telling if images are related to the specified moment in text and therefore able to retrieve the pictures accordingly.}

     \TEXT{}
     {In the ImageCLEF LMRT 2020 subtask the proposed automatic retrieval system achieved a score of 0.03 in the F1-measure@10 evaluation methodology. Even though this scores are not competitve when compared to other teams systems scores, the built system presents a good baseline for future work. }

\EndTitlePage
\titlepage\ \endtitlepage % empty page
