\cleardoublepage

\chapter{Introduction}
This chapter gives an introduction to the surrounding theme addressed in this thesis. In that sense, firstly the contextualization of the theme and the respective motivation will be presented. The different challenges, the objectives that are intended to be reach and the contributions given to the community are also described. Finally the document structure and organization is explained.


\section{Context}
The pervasive creation and consumption of content, especially visual content, is ingrained into our modern world. In the past, the main purpose given to pictures was to save moments of past events. People nowadays are constantly consuming visual media content. Pictures, images and photos have many different usages, they can either be used to save a moment or to document processes; we use them in engineering, in art, in science, in medicine, in entertainment and also in advertising. \cite{Zhang2008}


With the rapid development of Internet of things (IOT) this growth in consumption of visual media content has increased the usage of wearable and smart technologies making the subject of lifelogging more prevalent in the recent years. Lifelogging is the task of tracking and recording personal data created trough the activities and behaviour of individuals during their day-to-day life in the form of images, video, biometric data, location and other data. The name given to this  dataset  is "lifelog data" and is rich in resources for contextual information retrieval. \cite{Ribeiro}

Some great examples of the usefulness of lifelogging is using it as memory extension for people who suffer from memory impairments such as Alzheimers, to find lost items during the day or even to understand human behaviour.



Most of the technical problems associated with creating, compressing, storing,transmitting, rendering and protecting image data are already solved. However we still face two main challenges which are the issues associated with image location and the continuous growth of image data (big data).

Locating images involves analysing them to determine their content, classifying them into related groupings, and searching for  images. In order to solve these problems, the current technology relies heavily on the image description, usually called as "image metadata". This data can either be captured automatically at creation time or manually added afterwards.

In the present time the development in the area of content–based analysis (indexing and searching of visual media) is increasing, this is where most of the research in image management is concentrated. Automatic analysis of the content of images, which in turn would open the door to content–based indexing, classification and retrieval, is an inherently difficult problem and therefore progress is slow. \cite{Zhang2008}





\section{Motivation}


%  Para esta secção os artigos do imageclef tem ideias porreiras.

\section{Challenges}

\section{Objectives}

\section{Contributions}

\section{Document Structure}