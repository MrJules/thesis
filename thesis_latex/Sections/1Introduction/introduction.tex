\cleardoublepage

\chapter{Introduction}
\label{ch:introduction}
This chapter gives an introduction to the surrounding theme addressed in this thesis. In that sense, firstly the contextualization of the theme and the respective motivation will be presented. The different challenges, the objectives that are intended to be reach and the contributions given to the community are also described. Finally the document structure and organization is explained.


\section{Context and Motivation}

The pervasive creation and consumption of visual media content is ingrained into our modern world. In the past, the main purpose given to pictures was to save moments of events. Nowadays people are constantly consuming visual media content. Pictures, images and photos have many different usages, not only we use them for social media but also we use them in engineering, in art, in science, in medicine, in entertainment and also in advertising \cite{Zhang2008}.



With the rapid development of Internet of things (IOT) this growth in consumption of visual media content has increased the usage of wearable and smart technologies making the subject of lifelogging more prevalent in the recent years. Lifelogging is the task of tracking and recording personal data created trough the activities and behaviour of individuals during their day-to-day life in the form of images, video, biometric data, location and other data. The name given to the data created by lifelogging has the name of "lifelog data" and it is rich in resources for contextual information retrieval \cite{Ribeiro}.

Some great examples of the usefulness of lifelogging is using it as memory extension for people who suffer from memory impairments such as Alzheimers, to find lost items during the day or even to understand human behaviour.

The problems related to creating, compressing, storing, transmitting, rendering and protecting image data are already solved. However there still exists two difficult problems to tackle which are the issues associated with image location and the continuous growth of image data (big data) \cite{Zhang2008}.

"Locating images involves analysing them to determine their content, classifying them into related groupings, and searching for  images. In order to solve these problems, the current technology relies heavily on the image description" \cite{Zhang2008}, usually called as "image metadata". This data can either be added automatically at the capturing time or manually added by someone afterwards.

According to the literature \cite{Zhang2008} : "In the present time the development in the area of content–based analysis (indexing and searching of visual media) is increasing, this is where most of the research in image management is concentrated. Automatic analysis of the content of images, which in turn would open the door to content–based indexing, classification and retrieval, is an inherently difficult problem and therefore progress is slow." 

However, if one day a fully automatic image/video retrieval system is implemented it will vastly improve the life quality of the human kind. A great example that we can apply at the present time is that it will be possible to backtrack the last few days of humans infected with COVID-19 through their lifelog data, which in turn would help to identify more possible infected and warn more people to get tested.

\section{Challenges}

As it has been described earlier in the chapter, creating an automatic system capable of fully analysing the content of images is a difficult problem. This difficulty comes from two main challenges which are image processing and text processing. 

Creating an automatic system capable of image retrieval means that the computer has to be able to understand images and text while at the same time being capable of relating both.

For the image processing challenge, the computer has to be able to extract relevant information from images like colors, objects, places, locations, indoor or outdoors, activities happening in the photo, people, etc. However in order to do this,  many different algorithms have to be implemented like object detection, activity recognition, scene recognition and others. The usage of so many algorithms can require extreme computational time and resources depending on the size of the dataset to be analysed. If one image requires 1 second to be fully processed, a dataset of 200.000 images of the same resolution would require approximately 2 days.

In order to teach the computer to understand text there has to be an underlying understanding of the language. Natural Language Processing algorithms have to be implemented, however these algorithms only allow the computer to extract the basics like which words are verbs, adjectives, nouns, etc. With this in mind, another algorithm has to be fully written from scratch capable of using that data to extract activities, locations, people, dates, indoor or outdoor, etc from a corpus of text.

Finally, the computer has to be capable of comparing the extracted features from the text with extracted features from the images, in order to associate images to text.

\section{Objectives}

The main objective of this work is to develop an automatic image retrieval system capable of participating in the ImageCLEF LMRT-subtask challenge (described in chapter \ref{ch:imageclef}). In order to achieve this objective the following tasks have to be accomplished:

\begin{itemize}
    \item Study on the state of the art of image processing and text processing algorithms.
    \item Choice of the main algorithms to be used.
    \item Code of an algorithm capable of processing images and extract relevant features.
    \item Code of an algorithm capable of processing text and extract relevant features.
    \item Code of an algorithm able to compare the extracted features from the images with the extracted features from the text and capable of associating images to text.
    \item Code of an algorithm capable of calculating the F1@score of the final results in order to test the system.
    \item Code of a batch script in order to make the system run with one click in order to facilitate the process.
  \end{itemize}


\section{Contributions}

Since the process of automatic image retrieval is still a complex problem this works aims at contributing with a baseline system for future investigations with some suggestions on how to improve it further. Additionally a study of the available technology is conducted that may help on finding new and better paths for future investigations on automatic image retrieval.


\section{Document Structure}
This document has a total of 8 chapters that are divided accordingly:




\begin{itemize}
  \item Chapter \ref{ch:introduction} presents the context and motivation along with the challenges and objectives.
  \item Chapter \ref{ch:imageclef} discusses the imageCLEF challenge.
  \item Chapter \ref{ch:computervision} provides a survey on the subject of feature extraction from images and video.
  \item Chapter \ref{ch:nlp} addresses the thematic of natural language processing and word embeddings.
  \item Chapter \ref{ch:initial_work} provides an overview on how the image processing stage of the automatic system was built.
  \item Chapter \ref{ch:text_stage} explains how the system is capable of word extraction and categorization and how the system is capable of image retrieval.
  
  \item Chapter \ref{ch:results} presents the achieved results in the imageclef challenge.
  \item Chapter \ref{ch:conclusions} describes the conclusions taken from the development of the work and provides some ideas for future work and investigations.
  
 
 
\end{itemize}

